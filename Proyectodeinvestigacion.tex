\documentclass{article}
\usepackage[utf8]{inputenc}

\title{Proyecto de investigación}
\author{Juan Daniel Gonzalez Puerta}
\date{Marzo 2020}

\begin{document}

\maketitle

\section*{Introducción}
Los ordenadores son aparatos muy prácticos. Tanto, que se han vuelto indispensables en el funcionamiento de una sociedad moderna. Fueron inventados para que ayudasen a aclarar una cuestión filosófica concerniente a los fundamentos de la matemática, el cual venia desde la antgueda con el concepto de infinito o los inconmensurables conducen a un trastorno lógico que estremece desde los cimientos de la Geometría griega, que siglos más adelante se retomaría y daría comienzo a la informatica. Comienza esta asombrosa historia con David Hilbert, un célebre matemático alemán, que a principios del siglo XX propuso la formalización completa de todo el razonamiento matemático. Pero resultó que era imposible formalizar el razonamiento matemático, por lo que, en cierto sentido, su idea fue un tremendo fracaso. Más, en otro sentido, tuvo un gran éxito, porque el formalismo ha sido uno de los grandes dones que nos ha hecho el siglo XX. No para el razonamiento o la deducción matemática, sino para la programación, para el cálculo, para la computación. Es por esto que la crisis de los fundamentos fue clave en la invención de la computación, sin embargo has más conceptos y personas que fueron clave entre ellos Bertrand Russell, Kurt Gödel y Alan Turing así como el estudio filosófico del infinito y paradojas. 

\section*{Desarrollo}
Bertrand Russell, matemático que al pasar el tiempo se tornaría filósofo, primero, y por último, humanista. Russell constituye una figura clave porque descubrió algunas paradojas muy perturbadoras en la lógica misma. Es decir, halló casos en los que razonamientos en apariencia impecables conducen a contradicciones. Las aportaciones de Russell fueron fundamentales para que se difundiese la idea de que estas contradicciones causaban una crisis grave y habían de ser resueltas de algún modo, por ejemplo la famosa paradoja de Russell.


La paradoja de Russell dice \textbf{“el conjunto de todos los conjuntos que no son miembros de sí mismos”}, se obtiene el resultado que declara que no puede existir. Esto se debe a que nos encontramos frente a una situación paradójica, puesto que si no pertenece a sí mismo, entonces es de la especie de conjuntos que no pertenecen a sí mismos y, por tanto, debería pertenecer a sí mismo por la definición del conjunto por el que nos preguntamos. Por otro lado, si pertenece a sí mismo, entonces, por su propia definición, no puede pertenecer a sí mismo, puesto que sus miembros son precisamente aquellos conjuntos que no pertenecen a sí mismo. En definitiva, este conjunto pertenece a sí mismo si y sólo si no pertenece a sí mismo.

La idea de Hilbert consistía en crear para el razonamiento, para la deducción y para la matemática un lenguaje artificial perfecto. Hizo, por tanto, hincapié en la importancia del método axiomático, donde se parte de un conjunto de postulados básicos (axiomas) y reglas bien definidas para efectuar deducciones y derivar teoremas válidos. La idea de trabajar matemáticamente de este modo se remonta a los antiguos griegos, y en particular, a Euclides y su geometría, un sistema de hermosa claridad matemática.

El interés de Hilbert era la construcción axiomática; consistente y completa de la totalidad de las matemáticas, seleccionando como punto de partida los números naturales y asumiendo que mediante el uso de axiomas se obvia la necesidad de definir los objetos básicos con el fin de lograr un sistema completo y consistente. 
Gödel dinamitó la visión de Hilbert en 1931. Hilbert estaba totalmente equivocado; no hay modo de que exista un sistema axiomático para la totalidad de la matemática en el que quede claro como el agua si un enunciado es verdadero o no. Con mayor precisión: Gödel descubrió que el plan falla aun limitándose a la aritmética elemental, es decir, a los números 0, 1, 2, 3...; La adición y la multiplicación. 

La demostración de la incompletitud dada por Gödel es muy ingeniosa. Muy paradójica. Gödel empieza, en efecto, con la paradoja del mentiroso: \textbf{“soy falsa”}, que no es ni verdadera ni falsa. En realidad, lo que Gödel hace es construir una aseveración que dice de sí misma: \textbf{“Soy indemostrable”}. Desde luego, hará falta muchísimo ingenio para poder construir en la teoría elemental de números, en la aritmética, un enunciado matemático que se describa a sí mismo y refleje la frase anterior, pero si fuéramos capaces de lograrlo, enseguida comprenderíamos que estaríamos en un aprieto. Porque si el enunciado es demostrable, entonces es necesariamente falso; estaríamos demostrando resultados falsos. Por otro lado si es indemostrable, como dice de sí mismo, entonces es verdadero, y la matemática, incompleta en ambos casos.

El siguiente avance de importancia tuvo lugar cinco años después, en Inglaterra, cuando Alan Turing descubrió la no-compatibilidad. Recordemos que, según Hilbert, debía existir “un procedimiento mecánico” que decidiese si una demostración se atenía a las reglas o no. Hilbert no aclaró nunca qué entendía por procedimiento mecánico. Turing, en esencia, vino a decir que se trataba de una máquina.


Turing encuentra inmediatamente un problema que ninguna máquina de las que llevan su nombre podría resolver: el problema de la detención, es decir, decidir de antemano si una máquina de Turing (o un programa de ordenador) acabará por hallar su solución deseada y, por tanto, se detendrá.

En 1936 publicó el artículo \textbf{“Sobre números computables, con una aplicación al Entscheidungsproblem”} (traducible como “problema de decisión”), que resultó ser el origen de la informática teórica. En él definía qué era computable y qué no lo era. Lo computable era todo aquello que podía resolverse con un algoritmo (conjunto de instrucciones finito que, mediante pasos sucesivos, lleva a la solución de un problema). El resto eran tareas no computables.

\section*{Conclusion}
cuando se piensa en formalizar una ciencia, se cree que solo basta con poner reglas y axiomas que sigan una especie de receta para poder tener una estructura clara y uniforme, sin embargo no siempre se pone en tela de juicio las bases de la ciencia misma y que antes de formalizarla esta ya presenta ciertas contradicciones o paradojas que son intrínsecas.

\hspace{2ex}

Como se evidencio el problema de la contradicción, de los bucles infinitos y la determinación son conceptos fundamentales para la computación y para las matemáticas, debido a los fallos, problemas e incertidumbres que estos generan si no son controlados.

\hspace{2ex}

Al igual que en la física, la biología y las matemáticas las crisis que estas ramas de las ciencias ha  tenido a lo largo de su historia ponen en tela de juicio los pilares fundamentales que las conforman, aunque el problema en un comienzo no tenía nada que ver con la computación, la solución si lo tenía, ya que esta crisis fue el motor para que Alan Turing creara una máquina que diera la solución a una parte del problema y de manera homologa dar el origen a la informática teórica.

\section*{Referencias}

\small 

\begin{verbatim}
 
[1] Bertrand Russell. (2020, 11 de marzo). Wikipedia, La enciclopedia 
    libre.Fecha de consulta: 18:22, marzo 27, 2020 desde https://es.w
    ikipedia.org/w/index.php?title=Bertrand_Russell&oldid=24187108.
    
[2] Horsten, Leon. «Philosophy of Mathematics». Edward N. Zalta, ed. Sta
    nford Encyclopedia of Philosophy (en inglés) (Fall 2008 Edition).
    
[3] David Hilbert. (2020, 5 de enero). Wikipedia, La enciclopedia libre.
    Fecha de consulta: 17:57, marzo27,2020 desde https://es.wikipedia.o
    rg/w/index.php?title=David_Hilbert&oldid=122517918
    
[4] K Gödel: “Los conceptos tienen una existencia objetiva” en My philo
    sophical viewpoint
    
[5] Gustavo Fernandez D: "SEMINARIO DE LOGICA Y FILOSOFIA DE LA CIENCIA 
    I, página 101: Desarrollos posteriores de intuicionismo y construc-    
    tivismo.
\end{verbatim}


\end{document}
